\chapter{LHC und LHCb-Experiment}
Die grundlegenden Gestzmäßigkeiten der Teilchenphysik werden durch das Standardmodell beschrieben. 



Um diese Theorie zu bestätigen werden vor allem Experimente an Teilchenbeschleunigern durchgeführt.
Der derzeit größte seiner Art befindet sich am europäischen Kernforschungzentrum (CERN\footnote{Organisation Européene pour la Recherche Nucléaire})
In diser Anlage werden Hadronen in zwei gegeläufigen Beschleunigerringen mit einem Unfang von \SI{27}{\kilo\meter} Hadronen beschleunigt und an XX 
Punkten zur Kollision gebracht. An diesen Orten werden die erzeugten Teilchen von den verschiedenen Experimenten des LHC mit verschiedenen 
Detektoren untersucht. Abbildung \ref{fig:lhc_overview} zeigt den Aufbau der Anlage mit den Experimenten.

\noindent
Eines dieser Detektoren ist das so genannte LHCb\footnote{Large Hadron Collider Beauty}-Experiment. Ziel dieser Anlage ist die Durchführung
von Messungen um Bereich der CP-Verletzung sowie die Suche nach neuer Physik.
In den für diese Messungen intressanten Kollisionen enstehen vor allem schwere
b-Quraks. Diese Teilchen enstehen dabei bevorzugt in einem kleinen
Winkelbereich entlang und entgegen der Strahlachse emmitiert. Aus diesem Grund
wurde der LHCb-Detektor als Vorwärtsspektrometer ausgelegt. Das bedeutet das
nur Teilchen in einem kleinen Raumwinkelbereich in Strahlrichtung detektiert
werden.
In Abbildung \ref(fig:lhcb_overview} ist der Aufbau des Experiments mit den
verschiedenen Subdetektoren dargestellt. Diese sollen im Folgenden kurz
vorgestellt werden. 
\paragraph{Vertex Locator}
\paragraph{RICH Detector}

