\chapter{LHC und LHCb-Experiment}
Die grundlegenden Gestzmäßigkeiten der Teilchenphysik werden durch das Standardmodell beschrieben. 



Um diese Theorie zu bestätigen werden vor allem Experimente an Teilchenbeschleunigern durchgeführt.
Der derzeit größte seiner Art befindet sich am europäischen Kernforschungzentrum (CERN\footnote{Organisation Européene pour la Recherche Nucléaire})
In diser Anlage werden Hadronen in zwei gegeläufigen Beschleunigerringen mit einem Unfang von \SI{27}{\kilo\meter} Hadronen beschleunigt und an XX 
Punkten zur Kollision gebracht. An diesen Orten werden die erzeugten Teilchen von den verschiedenen Experimenten des LHC mit verschiedenen 
Detektoren untersucht. Abbildung \ref{fig:lhc_overview} zeigt den Aufbau der Anlage mit den Experimenten.

\noindent
Eines dieser Detektoren ist das so genannte LHCb\footnote{Large Hadron Collider Beauty}-Experiment. Ziel dieser Anlage ist die Durchführung
von Messungen um Bereich der CP-Verletzung sowie die Suche nach neuer Physik. 
