\chapter{LHC und LHCb-Experiment}
Einer der weltweilt leistungsfähisgten Teilchenbeschleuniger befindet sich am
europäischen Kernforschungszentrum (CERN\footnote{Council du ...}) in der Nähe
von Genf. Im Large Hadron Collider werden in einem \SI{}{\kilo\meter} langen
Tunnel Protonen und Bleiionen auf \SI{}{\percent} der Lichtgeschwindigkeit
beschleunigt und an XX verschiedenen Orten zur Kollision gebracht. 
Die Teilchenn durchlaufen hierzu als erstes Linearbeschleuniger (LINAC) und
dann mehrere Vorbeschleuniger. Danach werden diese in die beiden Speicherringe
des LHC eingespeist. Diese verlaufen im Tunnel des ehemaligen Large
Electron-Positron Colliders (LEP) in \SI{45}{\meter} bis
\SI{70}{\meter}\cite{evans}
Tiefe. In diesen Ringen werden die Protonen (bzw. Bleionen) mit Hilfe von
elektrischen Feldern in entgegengesetzter Richtung  auf Energien von jeweils
\SI{}{\terra\eV} beschleunigt. An den Interaktionspunkten kommt es dann zu
Kollisionen mit einer Schwerpunktsenergie von \SI{\terra\eV}. Diese werden dann
von den verschiedenen Expermenten des LHC untersucht. Abbildung
\ref{f:lhc_overview} zeigt schematisch den Aufbau der Anlage. In der Abbildung
ist auch die Lage der vier wichtigsten Experimente (ATLAS, CMS, ALICE, LHCb) am
LHC dargestellt. Auf letzteres soll im Folgenden weiter eingegangen werden.
\noindent
Ziel des Large Hadron Collider Beauty (LHCb)-Experiments dieser Anlage ist die Durchführung
von Messungen im Bereich der CP-Verletzung sowie die Suche nach neuer Physik.
In den für diese Messungen intressanten Kollisionen enstehen vor allem schwere
b-Quraks. Diese Teilchen enstehen dabei bevorzugt in einem kleinen
Winkelbereich entlang und entgegen der Strahlachse emmitiert. Aus diesem Grund
wurde der LHCb-Detektor als Vorwärtsspektrometer ausgelegt. Das bedeutet das
nur Teilchen in einem kleinen Raumwinkelbereich in Strahlrichtung detektiert
werden.
In Abbildung \ref{fig:lhcb_overview} ist der aktuelle Aufbau des Experiments mit den
verschiedenen Subdetektoren dargestellt. Diese sollen im Folgenden kurz
vorgestellt werden. 
\paragraph{Vertex Locator (VELO)}
\paragraph{RICH Detector}
\paragraph{Tracking Stationen}
\paragraph{Kalorimeter}
\paragraph{Myonen Kammern}
\section{Updgrade}



%sagemathcloud={"latex_command":"cd ..; xelatex -synctex=1 -interact=nonstopmode 'thesis.tex'; ln thesis.pdf Inhalt/02_lhcb.pdf"}
